\chapter{Előismeretek}

\section{Metamodell}

Egy szakterület-specifikus nyelvet (Domain Specific Language, DSL) tipikusan egy metamodell (MM) és a hozzá tartozó jólformáltsági kényszerek (Well-Formedness Constraints, WF) adnak meg. A metamodell specifikálja a modell alapstruktúráját, továbbá meghatározza a címkézett gráf  éleinek illetve csúcsainak lehetséges típusait. A jólformáltásgi kényszerek tervezési szabályok hozzáadásával ezt tovább szűkítik.

\begin{definition}[Szignatúra]
Szignatúra alatt egy olyan $\Sigma = \langle\mathit{C}, \mathit{R}, \mathit{A}, \mathit{E}, \mathit{L}\rangle$ struktúrát értünk, ahol $C = \{C_1, \ldots, C_i\}$ unáris típus predikátumokat, $R = \{R_1, \ldots, R_j\}$ bináris referencia predikátumokat, $A = \{A_1, \ldots, A_k\}$ bináris attribútum predikátumokat, $E = \{E_1, \ldots, E_l\}$ unáris enumeráció (felsorolható típus) predikátumokat és $L = \{L_1, \ldots, L_m\}$ az enumerációkat alkotó literálokat jelölnek.
\end{definition}

% \begin{definition}[Modellezési nyelv]
% ---
% \end{definition}

\begin{definition}[Metamodell]
Egy metamodell meghatározza egy modellezési nyelv legfőbb típusait, referenciáit, attribútumait és alapvető struktúráját. Formálisan: a metamodell meghatároz egy $\Sigma = \langle\mathit{C}, \mathit{R}, \mathit{A}, \mathit{E}, \mathit{L}\rangle$ szignatúrát, az $E = \{E_1, \ldots, E_l\}$ enumerációkhoz tartozó literálokat és az alábbi strukturális kényszereket:
\begin{itemize}
  \item $ \mathit{\textbf{MUL}}^\mathit{up}:  \{R_1, \ldots, R_j \} \rightarrow \mathbb{N}^+ \cup  \{*\} $ \\
  	Felső számosság (Upper Multiplicity): egy felső korlátot szab meg a referencia számosságát illetően, azaz megadja, hány ilyen referencia-példány lehet két osztály között.
  	A $ * $ jelölés korlátlan felső határt jelent.

  
  \item $ \mathit{\textbf{MUL}}^\mathit{low}:  \{R_1, \ldots, R_j \} \rightarrow \mathbb{N} $ \\
  Alsó számosság (Lower Multiplicity): egy alsó korlátot szab meg a referencia számosságát illetően.
  
  \item $ \mathit{\textbf{CON}}: \{R_1, \ldots, R_j \} \rightarrow \{0, 1\} $ \\
  	Tartalmazási él (Containment Reference): meghatározzam hogy az adott él tartalmazási él-e.
  	Az EMF példánymodellek szigorú tartalmazási hierarchia szerint épülnek fel, egy irányított fa struktúraként.
	
  \item $ \mathit{\textbf{SUP}}: \{C_1, \ldots, C_i\} \times \{C_1, \ldots, C_i\} \rightarrow \{0, 1\} $ \\
	Ősosztály (Superclass): általánosítás meglétét adja meg két osztály között: egy konkrétabb (gyerek) osztály rendelkezik az általánosabb (szülő) osztály minden tulajdonságával.
	\todo{"öröklés, leszármazás"  definiálása?},
	Az EMF-ben támogatott a többszörös öröklés, de ezek az osztályok nem képezhetnek hurkot.
  \item $ \mathit{\textbf{ABS}}: \{C_1, \ldots, C_i \} \rightarrow \{0, 1\} $ \\
	Absztrakció: meghatározza, hogy az adott osztály absztrakt-e. Az absztrakt osztályok nem példányosíthatók, a belőlük leszármaztatott nem-absztrakt osztályok viszont igen.

  \item $ \mathit{\textbf{INV}}: \{R_1, \ldots, R_j \} \times \{R_1, \ldots, R_j\} \rightarrow \{0, 1\} $
  \item $ \mathit{\textbf{TCR}}^\mathit{src}: \{R_1, \ldots, R_j \} \rightarrow \{C_1, \ldots, C_i\} $  
  \item $ \mathit{\textbf{TCR}}^\mathit{trg}: \{R_1, \ldots, R_j \} \rightarrow \{C_1, \ldots, C_i\} $
  \item $ \mathit{\textbf{TCA}}^\mathit{src}: \{A_1, \ldots, A_k\} \rightarrow \{C_1, \ldots, C_i\} $
  \item $ \mathit{\textbf{TCA}}^\mathit{trg}: \{A_1, \ldots, A_k\} \rightarrow \{\mathit{Int, String, Real, Boolean, Literal}\}$

 
\end{itemize}

\end{definition}


\section{Példánymodell}
\begin{definition}[Példánymodell]
Példánymodell (Instance Model, IM) alatt egy olyan $M = \langle Obj_M, \mathcal{I}_M \rangle$ logikai struktúrát értünk, ahol $Obj_M $ egy véges, halmaza a modellben szereplő egyedeknek, $\mathcal{I}_M$ pedig interpetációt ad a szignatúrákban szereplő összes predikátumhoz, a következőképpen:
\begin{itemize}
\item[--] $\mathcal{I}_M(C_i) : Obj_M \rightarrow \{0, 1\}$
\item[--] $\mathcal{I}_M(R_j) : Obj_M \times Obj_M \rightarrow \{0, 1\}$
\end{itemize}
Továbbá a halmazban szereplő összes egyedről tudjuk, hogy létezésük biztos, azaz 1.
\end{definition}

\section{Parciális modell}

A parciális - vagy más néven részleges - modellekben (Partial Model, PM) megjelennek a bizonytalan alkotórészek is.
Ezek a modellek még befejezetlenek, olyan elemek is megjelennek bennük, amiknek a létezése még nem ismert, ezeket egyketteddel jelöljük.

\begin{definition}[Parciális modell]
Parciális modell alatt egy olyan logikai struktúrát értünk, ahol\ldots
\end{definition}