%----------------------------------------------------------------------------
\chapter{\bevezetes}
%----------------------------------------------------------------------------

\section{Kontextus}
Kritikus rendszerek tervezésére egyre szélesebb körben alkalmaznak modellezőeszközöket.
Ezek az eszközök gráf-alapú modelleket felhasználva képesek javítani a fejlesztési folyamat produktivitását.
A modellezőeszközök működésének biztosítása érdekében magukat a modellezőeszközöket is tesztelni és teljesítménymérni kell, ehhez pedig automatikusan generált gráfokra van szükség.

\todo{[WHR14] hivatkozas}

\section{Problémafelvetés}
A modellezőeszközök teszteléséhez generált gráfoknak realisztikusnak és diverznek kell lenniük.
Előfordulhat, hogy adott környezetben nem ismerjük a referenciaként szolgáló konkrét modellt


modellezőeszközöket teljesítménymérni és tesztelni kell,
realisztikus modellek kellenek hozzá (titkosítás)

\section{Célkitűzés}

\section{Kontribúció}

\section{Hozzáadott érték}

\section{A dolgozat felépítése}
Dolgozatom második fejezetében összefoglalom a modellgeneráláshoz 
\todo[inline]{modellgeneráláshoz?} szükséges előismereteket.
A harmadik fejezetben egy összefogó képet...
A negyedik fejezetben...
Az ötödik fejezetben...
A hatodik fejezetben...




